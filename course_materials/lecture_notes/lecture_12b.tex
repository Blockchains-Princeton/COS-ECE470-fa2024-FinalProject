\documentclass{article}
\usepackage[utf8]{inputenc}
\usepackage[margin=1.25in]{geometry}
\usepackage[english]{babel}
\usepackage{listings}
\usepackage{fancyhdr}
\usepackage{makecell}
\usepackage{threeparttable}
\usepackage{subfig}
\usepackage{graphicx}  
\newsavebox{\measurebox}
\pagestyle{plain}
\fancyhf{}
\usepackage{tabularx} 
\usepackage{multirow}
\usepackage{url}
\usepackage{amsmath}
\usepackage{amssymb}
%\usepackage{hyperref}
\usepackage{markdown}
\usepackage{amsthm}
\usepackage{caption}
%\usepackage{subcaption}

\newcommand{\indicator}{{\bf 1}}
\newcommand{\pramod}[1]{{\color{red}
\footnotesize[Pramod: #1] }}
\newcommand{\xw}[1]{{\color{green}
\footnotesize[Xuechao: #1] }}
%\let\proof\relax
%\let\endproof\relax 



\newcommand{\POM}{$\mathsf{POM}\ $}
\usepackage{float}
\usepackage{amsmath}
\usepackage[colorlinks=true, allcolors=blue]{hyperref}
\graphicspath{ {./figures/} }
\newcommand{\h}{\textsf{H}}
\newtheorem{lemma}{Lemma}[]

\rhead{Principles of Blockchains}
\lhead{Lecture 13}

\title{Lecture 13:   Layer 2 Scaling: Payment Channels}
\author{
Principles of Blockchains, Princeton University, \\ 
Professor: Pramod Viswanath \\ 
Scribe: Ranvir Rana\\
}
%\date{March 4, 2021}
\date{\today}

\begin{document}
\maketitle

\begin{abstract}
    So far, we have considered scaling methods that alter the consensus protocol itself (albeit in modest ways). In this lecture, we study how to scale existing blockchain performance {\em without} changing the consensus layer. This is done  by extracting trust from the  consensus embedded in the core blockchain, but offloading computation and storage. Given that the consensus protocol (layer 1) is untouched, these methods are known to afford {\em Layer 2} scaling. Two prominent instances are  {\em payment channels} (focused on the UTXO state management system) and {\em side blockchains} (can handle account based systems and smart contracts). This lecture studies the former of these methods.
\end{abstract}



\section*{Introduction}
In the last three lectures, we have considered proposals to scale throughput, latency, storage, communication and computation. In each of these proposals, the longest chain consensus protocol was modified. In a practical blockchain already operating in the real world, it is quite onerous to change the consensus layer: such a change would require a ``meta consensus" among the participating nodes, i.e., consensus on how to change the consensus mechanism! Even a successful switch in the consensus mechanism would still lead to a ``hard fork" in the ledger where the two paths would be following different consensus protocols. In this context, proposals to scale performance without changing the consensus layer are very appealing. Such are the goals of this lecture, where we discuss the two most promising ``layer 2"  proposals that scale performance by offloading computation and storage  without impacting the core technology of blockchains (``layer 1'') and overall security. The first mechanism is via {\em side chains} and allows a general account based model and smart contract. The second mechanism is via {\em payment channels} and is more restricted, e.g., to the UTXO model of handling payments.  



\section*{Payment Channels}
Payment channels support users to make payments for multiple times off chain while only submitting transactions on-chain to start the channel and handle a dispute. It locks parts of the state of the blockchain when starting a channel, processes transactions associated with this locked state in an application layer, and finally unlocks the channel with the updated state.

In a UTXO system like Bitcoin, a transaction contains a sequence of inputs and outputs. Outputs serve as {\em locks} for a specific amount of funds, while inputs serve as keys of corresponding outputs to unlock the funds and transfer them between accounts. There are three new types of cryptographic primitives that allow ``flexible locks", so that the trust can be extracted outside the blockchain. In particular there are three important cryptographic primitives we explore here.
\begin{itemize}
    \item {\sf Multisig}. the locking transaction needs to be signed by $k$ out of $n$ public keys
    \item {\sf Hashlock}. unlocking requires the owner's public key and a secret
    \item {\sf Timelock}. requirement is related to block height and other time-based conditions
\end{itemize}

%Generally, \textsf{KeyGen} $\rightarrow (sk, pk)$
%\pramod{Need to go through these 3 crypto primitives in detail, give directions to libraries and setup some MPs if possible. The idea is that these details will really help in the description of the payment channels below. }


According to the payment direction,  payment channels can be classified into three categories, one-way payment channels, two-way payment channels and payment networks. Here we discuss the first two types. %\pramod{if things are getting complicated, we can just focus on the first type. The important thing is to explain in as much detail as needed to actually code up a prototype. The lecture notes are a form of engineering spec, not a Wiki article. :-)}

\subsection*{One-way payment channels}
One-way payment channels  only allow funds to flow in one direction, from payer to recipient. For example, suppose Alice wants to pay Bob in increments of 0.1 BTC up to a maximum of 1 BTC. Each of the payments can be made ``on-chain", however the payment channel proceeds in the following steps (see Figure~\ref{fig:channel}). 
\begin{enumerate}
    \item \textbf{Creating the channel.} Alice signs a {\em funding transaction} and posts it on the blockchain to create the channel. The funding transaction contains a single input with 1 BTC that is signed by Alice, the output can either: 
    \begin{itemize}
        \item contain an address that is derived from both Alice and Bob's public keys (\textsf{multiaddr = GetAddrbyAccount(pk$_1$, pk$_2$)}). And it can only be unlocked with both Alice and Bob's secret keys, neither one can unlock it alone. 
        %\peiyao{The sentence should give such information: there exist a function we can use to generate address given pubkey, the unlock part will be explained below in closing the channel point.}
        % \pramod{can you talk a bit about how the multisig is executed, whether it is off-chain or on-chain and how the off chain is reflected onto the on-chain transaction?}
        \item contain Alice's address, and a timelock that specifies the expiration time of the channel.  %\pramod{there is something weird about this part -- we are talking about how Alice pays Bob, but you are only covering what happens if Bob is not paid and how Alice recovers her money} \peiyao{This is the second branch. 
        If Bob is online, he can receive the payment through posting a closing transaction, otherwise the coins return to Alice after the time expires. 
    \end{itemize}
    \item \textbf{Updating the payments.} After creating the channel, Alice can make payments to Bob and each payment creates an intermediate off-chain transaction, which is called {\em commitment transaction}. A commitment transaction uses the output of funding transaction as input, and specifies how the funds held by Alice are sent to Bob through the output, e.g. 0.9 BTC to Alice, 0.1 BTC to Bob. Alice can continue paying Bob through the payment channel in this fashion. Each time she wants to pay another 0.1 BTC to Bob, she constructs a new commitment transaction from the same funding transaction output and sends it to Bob. Those commitment transactions are held by Bob, who can later decide to post any one (and only one) of them to the blockchain to receive the funds.
    \item \textbf{Closing the channel.}  To close the channel, a {\em closing transaction} is generated and posted on blockchain to close the channel and update the state. There are two possible ways to close the channel corresponding to the funding transaction,
    \begin{itemize}
        \item Cooperative: Bob posts a commitment transaction, whose input contains a signature signed with both Alice and Bob's secret keys (2-of-2 {\sf multisig}). The signature will be verified on-chain and once it matches the output address of a funding transaction, Alice successfully makes the payments.  %\peiyao{this paragraph corresponds to two functions in MP, sig = sign(sk1, sk2) and verify(multiaddress, sig)}
        \item Non-cooperative: Bob is offline and channel expires (e.g. one-day {\sf timelock}), Alice posts the closing transaction on chain whose input contains Alice's signature and an expired timelock, Alice recovers her money.
    \end{itemize}

\end{enumerate}

\begin{figure}
    \centering
    \includegraphics[width=0.6\textwidth]{figures/channel.png}
    \caption{A one-way payment channel example.}
    \label{fig:channel}
\end{figure}
\subsection*{Two-way payment channels}
Two-way payment channels enable two parties to transfer funds to each other. Suppose now Alice and Bob both have 0.5 BTC to make payments. Consider the following steps (see Figure~\ref{fig:2channel}). 
\begin{enumerate}
    \item \textbf{Creating the channel.} The funding transaction now contains two inputs, each with 0.5 BTC, the first is signed by Alice and the second is signed by Bob. The output contains 1 BTC signed by both Alice and Bob using 2-of-2 {\sf multisig}. Before starting making payments, Alice and Bob will each create a secret and exchange the hashes of the secrets with each other.  The opening transaction will not be signed and posted on-chain until Alice and Bob receive a special commitment transaction respectively. The commitment transaction  here will contain an input with 1 BTC signed by both Alice and Bob (the output of a funding transaction), and two outputs. For the transaction that is held by Alice, the outputs are (1) 0.5 BTC to Bob, and (2) 0.5 BTC with timelock to Alice or to Bob if he knows Alice's secret, these outputs are signed by Bob. Similar for Bob, the outputs are signed by Alice and contain (1) 0.5 BTC to Alice, and (2) 0.5 BTC with one-week timelock to Bob or to Alice if she knows Bob's secret.
    \item \textbf{Updating the payments.} Alice and Bob start generating a normal commitment transaction. Suppose Alice wants to send Bob 0.1 BTC, then she will sign a transaction with outputs: (1) 0.4 BTC to Alice, and (2) 0.6 BTC with one-week timelock to Bob or to Alice if she knows Bob's secret. For each payment, Alice and Bob will generate a new secret and exchange the older secret to ensure that the older transaction will not be posted on chain.
    \item \textbf{Closing the channel.} When one of the parties is offline, the channel can be closed by revealing any one of the commitment transactions in a non-cooperative way. And in the cooperative situation, Alice and Bob can create a transaction sending the settled balance to each party.

\end{enumerate}




\begin{figure}
    \centering
    \includegraphics[width=0.6\textwidth]{figures/twoway.png}
    \caption{A two-way payment channel example.}
    \label{fig:2channel}
\end{figure}


\subsection*{Multi-hop payment channels}




\end{document}